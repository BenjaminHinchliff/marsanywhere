\chapter{Implementation} \label{sec:implementation}

Since no existing suitable datasets existed prior to this work, we first create
a dataset for use on this problem.

\section{Data Sources}

% \begin{itemize}
%   \item Satellite - JEZ GeoTiff from Mars Trek
%   \item DEM - JEZ Geotiff from HiRISE release \url{https://www.uahirise.org/ESP_059352_1985}
%   \item Ground Imagery - PDS data node beta archive explorer \url{https://pds-imaging.jpl.nasa.gov/beta/archive-explorer}
% \end{itemize}

Data was all sourced from publicly available NASA resources.

The Ground imagery was collected from NASA Jet Propulsion Labratory's Planetary
Data Systems data node, filtered for cylindrically projected mosaic panoramic
images released from the perseverance rover \cite{PDS_IMG_Node_Mars2020}.

The satellite views were sourced from processed GeoTIFF map data released by
NASA as part of the planning for landing the perseverance rover in the Jezero
crater region. This data was initially collected from the GeoTIFF released as a
part of the Mars Trek application \cite{MarsTrek_Law2017}, but was later swapped
to use data from the slightly less comprehensive Jezero crater region HiRISE
data release for consistency with the DEM data.

The DEM data was sourced from the HiRISE data release done by the USGS
\cite{USGS_Mars2020_TRN_HiRISE_2020}. 

\section{Data Processing Pipeline}

\subsection{Ground Imagery}

% \begin{itemize}
%   \item VICAR Images
%   \item Normalization
%   \item Gamma correction
%   \item Reprojection
%   \item Resizing
% \end{itemize}

The Perseverance ground imagery's native form is in VICAR format, and so imagery
was opened using a patched version rms-vicar, a python-based VICAR file reader
maintrained by SETI \cite{hinchliff2025rms‐vicar}. While PNG releases of
converted images are also done, VICAR was chosen as the imagery sounce since the
specifics of the imagery conversion and if any data is lost is unknown to us.

Since imagery contained absolute intensity instead of sRGB image data, it was
first normalized and then gamma corrected ($\gamma = 2.2$), for usage as
standard 24-bit PNG image data.

The associated PDS label data in the VICAR images was extracted and stored in a
metadata csv file, for later usage in processing the satellite imagery.

Since the mosaic imagery often had different extents (sometimes being only a
small portion of a much larger image), it needs to be reporjected to a standard
panoramic form for usage in machine learning. This was done using a Python and
OpenCV to build a remap operation to shift the view and height to approximate a
uniform perspective for the imagery. (maybe go into laborious detail on this?)

Lastly, the images were resized to be 1024x512, an arbitrary size comparable to
other works done in the field.

\subsection{Satellite Imagery}

% \begin{itemize}
%   \item HiRISE Geotiff
%   \item SPICE localization lookup
%   \item GDAL tile cropping
%   \item Blender ground view reprojection
% \end{itemize}

The ground imagery does not contain localization information trivially
convertible to latitude and longitude. Position information is given in a
mars-fixed Cartesian frame relative to a recursively defined "site frame", where 
the first site frame is defined relative to the rover landing location and the
second site frame is defined relative to the first, etc.

We sidestep this complexity by instead locating the rover based on the time the
image was assembled into a mosaic panorama and looking up it's position in
public releases of rover localizations. In our work, we used publicly available
SPICE data to do this \cite{mars2020_spice_kernels_2025}. We may have been able
to use the simpler PLACES database, but is has a lower temporal resolution of
only one position per day and no inbuilt interpolation. Using python and
Spiceypy, we looked up the latitude and longitude of the Perseverance rover at a
given time when a mosaic image was assembled \cite{SpiceyPy_Annex2020}. Since
most mosaics are assembled while the rover is still at the location of the
mosaic images, this provides a good heuristic for the rover's approximate
location.

For each located image, a 0.005\textdegree{} diameter tile of satellite image and DEM
data was extracted from the full data release using Python and GDAL
\cite{USGS_Mars2020_TRN_HiRISE_2020, GDAL_2025}. This comes out to a 1124 by 1186
tile of the satellite image data at 25 cm/pixel, and a 296 by 296 tile of the
DEM image data at 1 m/pixel. This makes the absolute extent of each tile just
under 300 m diameter.


\subsection{HuggingFace Dataset Conversion}

\section{ControlNet Model}

\subsection{Associative Masking}


