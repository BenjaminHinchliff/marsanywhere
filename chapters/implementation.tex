\chapter{Implementation} \label{sec:implementation}

Since no existing suitable datasets existed prior to this work, we first create
a dataset for use on this problem.

\section{Data Sources}

% \begin{itemize}
%   \item Satellite - JEZ GeoTiff from Mars Trek
%   \item DEM - JEZ Geotiff from HiRISE release \url{https://www.uahirise.org/ESP_059352_1985}
%   \item Ground Imagery - PDS data node beta archive explorer \url{https://pds-imaging.jpl.nasa.gov/beta/archive-explorer}
% \end{itemize}

Data was all sourced from publicly available NASA resources.

The Ground imagery was collected from NASA Jet Propulsion Labratory's Planetary
Data Systems data node, filtered for cylindrically projected mosaic panoramic
images released from the perseverance rover \cite{PDS_IMG_Node_Mars2020}.

The satellite views were sourced from processed GeoTIFF map data released by
NASA as part of the planning for landing the perseverance rover in the Jezero
crater region. This data was initially collected from the GeoTIFF released as a
part of the Mars Trek application \cite{MarsTrek_Law2017}, but was later swapped
to use data from the slightly less comprehensive Jezero crater region HiRISE
data release for consistency with the DEM data.

The DEM data was sourced from the HiRISE data release done by the USGS
\cite{USGS_Mars2020_TRN_HiRISE_2020}.

\section{Data Processing Pipeline}

\subsection{Ground Imagery}

% \begin{itemize}
%   \item VICAR Images
%   \item Normalization
%   \item Gamma correction
%   \item Reprojection
%   \item Resizing
% \end{itemize}

The Perseverance ground imagery's native form is in VICAR format, and so imagery
was opened using a patched version rms-vicar, a python-based VICAR file reader
maintrained by SETI \cite{hinchliff2025rms‐vicar}. While PNG releases of
converted images are also done, VICAR was chosen as the imagery sounce since the
specifics of the imagery conversion and if any data is lost is unknown to us.

Since imagery contained absolute intensity instead of sRGB image data, it was
first normalized and then gamma corrected ($\gamma = 2.2$), for usage as
standard 24-bit PNG image data.

The associated PDS label data in the VICAR images was extracted and stored in a
metadata csv file, for later usage in processing the satellite imagery.

Since the mosaic imagery often had different extents (sometimes being only a
small portion of a much larger image), it needs to be reporjected to a standard
panoramic form for usage in machine learning. This was done using a Python and
OpenCV to build a remap operation to shift the view and height to approximate a
uniform perspective for the imagery. (maybe go into laborious detail on this?)

Lastly, the images were resized to be 1024x512, an arbitrary size comparable to
other works done in the field.

\subsection{Satellite Imagery}

% \begin{itemize}
%   \item HiRISE Geotiff
%   \item SPICE localization lookup
%   \item GDAL tile cropping
%   \item Blender ground view reprojection
% \end{itemize}

The ground imagery does not contain localization information trivially
convertible to latitude and longitude. Position information is given in a
mars-fixed Cartesian frame relative to a recursively defined "site frame", where
the first site frame is defined relative to the rover landing location and the
second site frame is defined relative to the first, etc.

We sidestep this complexity by instead locating the rover based on the time the
image was assembled into a mosaic panorama and looking up it's position in
public releases of rover localizations. In our work, we used publicly available
SPICE data to do this \cite{mars2020_spice_kernels_2025}. We may have been able
to use the simpler PLACES database, but is has a lower temporal resolution of
only one position per day and no inbuilt interpolation. Using python and
Spiceypy, we looked up the latitude and longitude of the Perseverance rover at a
given time when a mosaic image was assembled \cite{SpiceyPy_Annex2020}. Since
most mosaics are assembled while the rover is still at the location of the
mosaic images, this provides a good heuristic for the rover's approximate
location.

For each located image, a 0.005\textdegree{} diameter tile of satellite image
and DEM data was extracted from the full data release using Python and GDAL
\cite{USGS_Mars2020_TRN_HiRISE_2020, GDAL_2025}. This comes out to a 1124 by
1186 tile of the satellite image data at 25 cm/pixel, and a 296 by 296 tile of
the DEM image data at 1 m/pixel. This makes the absolute extent of each tile
just under 300 m diameter.

\section{Ground View Reprojection}

Preliminary testing found that a Controlnet based model performed poorly on
sattelite/dem imagery alone. We believe this is likely because the U-Net
architecture used by Stable Diffusion 2.1 struggles to encode the quite complex
non-linear perspective shifts in the data, and as such stuggles to converge into
a useful result. This is consistent with the work of
\citeauthor{li2024crossviewdiff}, where they chose to use a ground view sky mask
to guide the diffusion process \cite{li2024crossviewdiff}. We do not have access
to sky mask data, and don't want to use it as a prior, so chose to instead use a
conceptually similar approach of reprojecting the sattelite view tiles into the
ground view space to make the persepctive shift minimal.

This was done using Blender, an open source 3d graphics suite that includes a
robust scripting API \cite{blender2025}. The sattelite view was projected onto a
plane of size 2.0 units. The plane was then subdivided in accordance with the
resolution of the ground DEM texture, and then the DEM texture was applied as a
displacement modifier to the plane. The plane was then offset by the scaled
height of the displacement height under the camera (the center of the texture),
such that the displaced plane would remain fixed near the scene origin. From
empirical testing, we found that a displacement strength of 0.1 and a camera height of 0.25 units produced resulsts reasonably consistent with the ground view imagery.

\begin{figure}
\begin{tikzpicture}[
  io/.style={
    draw=black
  },
  stage/.style={
    draw=black,
    minimum height=0.5\linewidth,
    minimum width=0.2\linewidth
  },
  descr/.style={
    fill=white,
    inner sep=2.5pt
  }
]

  \node [stage] (gdal) {
    \begin{minipage}{1.5cm} % adjust width for your layout
        \centering
        \rotatebox{270}{GDAL Tile Extraction}\\[2pt]
        \includegraphics[width=\linewidth]{gdalicon.png}
    \end{minipage}
  };

  \node [io, above left=-3.4cm and 1cm of gdal] (rgbfull) {
    \includegraphics[height=0.18\linewidth]{JEZ_hirise_soc_007_orthoMosaic_25cm_Ortho_blend120_thumb.png}
  };

  \node [io, below=of rgbfull] (dtmfull) {
    \includegraphics[height=0.18\linewidth]{JEZ_hirise_soc_006_DTM_MOLAtopography_DeltaGeoid_1m_Eqc_latTs0_lon0_blend40_thumb.png}
  };

  \node [io, above right=-3.4cm and 1cm of gdal] (rgbtile) {
    \includegraphics[height=0.18\linewidth]{N_LRGB_1252_RZS_0590000_CYL_L_AUTOGENJ03_sat.png}
  };

  \node [io, below=of rgbtile] (dtmtile) {
    \includegraphics[height=0.18\linewidth]{N_LRGB_1252_RZS_0590000_CYL_L_AUTOGENJ03_dem.png}
  };

  \node [io, above=of gdal] (ground) {
    \includegraphics[height=0.18\linewidth]{N_LRGB_1252_RZS_0590000_CYL_L_AUTOGENJ03.png}
  };

  \node [stage, below right=-3.4cm and 1cm of rgbtile] (blender) {
    \begin{minipage}{1.5cm} % adjust width for your layout
        \centering
        \rotatebox{270}{Blender Reprojection}\\[2pt]
        \includegraphics[width=\linewidth]{blender_icon_128x128.png}
    \end{minipage}
  };

  \node [io, below left=1cm and -3cm of blender] (vis) {
    \includegraphics[height=0.18\linewidth]{N_LRGB_1252_RZS_0590000_CYL_L_AUTOGENJ03_vis.png}
  };

  \draw[-stealth] (ground.south) to node[descr] {Rover Mosaic Creation Time} (gdal.north);
  \draw[-stealth] (rgbfull.east) |- ([yshift=2cm]gdal.west);
  \draw[-stealth] (dtmfull.east) |- ([yshift=-2.25cm]gdal.west);
  \draw[-stealth] ([yshift=2cm]gdal.east) |- (rgbtile.west);
  \draw[-stealth] ([yshift=-2.25cm]gdal.east) |- (dtmtile.west);
  \draw[-stealth] (rgbtile.east) |- ([yshift=2cm]blender.west);
  \draw[-stealth] (dtmtile.east) |- ([yshift=-2.25cm]blender.west);
  \draw[-stealth] (blender.south) -- ++(0,-0.5) -| (vis.north);

\end{tikzpicture}

\caption{Data Processing Pipeline}
\label{fig:data-pipeline}

\end{figure}

\subsection{HuggingFace Dataset Conversion}

For compatibility with the HuggingFace datasets library, a script was created to
allow for a custom loading of the dataset. This was first written using the
HuggingFace datasets api and python library for a custom scripted dataset, but
with the depreciation of custom scipts (for security reasons) in huggingface
datasets v4.0.0, the script was updated to use the imagefolder dataset loader.

This provides additional benefits in that the train and test partitions are
simple directories, allowing them to be more easily used for other problems such
as for calculating results.

\subsection{Train Test Split}

The dataset was split into two partitions, a train split and the holdout test split using a simple random sampling, with 80\% train split and 20\% test split.

\section{ControlNet Model}

A controlnet model based on Stable Diffusion 2.1 was trained on the projected
ground view imagery. This was trained using a script originally created to fine
tune stable diffusion 2.1 models, and was subsequently modified to automatically
split and optionally appropriately mask the train images. The model was trained
on the dataset with a learning rate of $lr = 10^{-5}$ for 2000 epochs on 2x
NVIDIA V100 GPU. To converve memory and allow for faster train, 16-bit floats
were used for the model.

\subsection{Associative Masking}

A factor of additional complexity with the training set is that the ground
imagery necessarily contains more data than the tile, much less the reprojected
tile, can provide.
